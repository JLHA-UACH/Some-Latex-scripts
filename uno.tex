%
%Este documento se puede usar como plantilla para un reporte sencillo, además de los formatos para lógica matemática
%
\documentclass[11pt]{article}

\usepackage[latin1]{inputenc}
\usepackage[T1]{fontenc}
\usepackage[spanish]{babel}

\usepackage{amssymb}
\usepackage{amsbsy}
\usepackage{amsmath}
\usepackage{latexsym}%,pstricks,color,epic,eepic}
\usepackage{multicol}
\usepackage{tikz-cd}
\usepackage[all]{xy}
\usetikzlibrary{cd}

\hoffset = -50pt
\voffset = -50pt
\textwidth = 500pt
\textheight = 615pt

\def\IR{{\rm I\!R}}  %for real numbers
\def\IN{{\rm I\!N}}  %for integer numbers
\def\IC{{\rm C}\llap{\vrule height7.1pt width1pt
     depth-.4pt\phantom t}} %for complex numbers \def\covD{{\rmI\!D}}
\def\IQ{{\rm Q}\llap{\vrule height7.7pt width1pt
     depth-.4pt\phantom t}} %for complex numbers \def\covD{{\rmI\!D}}

%   Si pones en tus definiciones las siguientes instrucciones:

\font\cmss=cmss10
\font\cmsss=cmss10 at 7pt

\def\IZ{\relax\ifmmode\mathchoice
{\hbox{\cmss Z\kern-.4em Z}}{\hbox{\cmss Z\kern-.4em Z}}
{\lower.9pt\hbox{\cmsss Z\kern-.4em Z}}
{\lower1.2pt\hbox{\cmsss Z\kern-.4em Z}}\else{\cmss Z\kern-.4em Z}\fi}

%   podr'as usar  $ \IZ $  para denotar los enteros.


%---------------*******************************------------------------

% Definition of title page:
\title{\textsf{
Aquí va el título
}}
\author{Poner tu nombre }
%\pagestyle{headings}
%\date{10 de octubre de 2006}

\begin{document}

  \maketitle



así es como decia en clase las citas se hacen así: bla, bla~\cite{3} bla, bla, bla
%\arrow[d,dash] 



$$
\begin{tikzcd}[column sep=small, row sep=tiny]
& P \ar[r, Rightarrow] & (P \ar[r, Rightarrow] & Q ) \\
& C \ar[drr, bend right = 95, dash]      & C \ar[r, bend right = 90, dash]  & C \\
&   							     &                  & \parbox{0.5cm}{\flushleft C} \\
&  &\parbox{0.5cm}{\flushright C} &
\end{tikzcd}
$$


ahora una mas grande aun no está completo:

$$
\begin{tikzcd}[column sep=small, row sep=tiny]
& ((P \parbox{0.5cm}{\centering$\wedge$} & Q)\ar[r, Rightarrow] & B) \ar[r,Rightarrow]   & (P \ar[r, Rightarrow] & (P \ar[r, Rightarrow] & Q )) \\
& C \ar[r, bend right = 90, dash] & C &  F & C \ar[dr, bend right = 95, dash]      & C \ar[r, bend right = 90, dash]  & F \\
& \parbox{0.5cm}{\flushright F} \ar[dr, bend right = 95, dash] &  & 									   &                  & \parbox{0.5cm}{\flushleft F} \\
& & F & &  &\parbox{0.5cm}{\flushright F} &
\end{tikzcd}
$$




\begin{thebibliography}{99}
\bibitem{1}  Calculo; Robert T. Smith, Roland B. Minton; McGraw Hill; 2000.
\bibitem{2}  Calculo diferencial e integral; Frank Ayres Jr:, Elliot Mendelson; McGraw Hill ;3ra Edicion.
\bibitem{3}  Calculo una variable; Thomas; Editorial Pearson; undécima edición.
\bibitem{4}  El calculo con geometría analítica; Louis Leithold; editorial Harla; 6ta edición; año 1992.
\end{thebibliography}

\end{document}
